% if you want the answers to appear uncomment the below
% \documentclass[answers]{exam}
% otherwise uncomment the below
\documentclass{exam}
\usepackage[bottom]{footmisc}
\usepackage{graphicx}
\usepackage[letterpaper, margin=.9in]{geometry}
\usepackage{natbib}
\bibpunct{(}{)}{;}{a}{,}{,}
\usepackage{url}
\def\UrlFont{\rm}


\usepackage{Sweave}

\usepackage[utf8x]{inputenc}
\usepackage{array}
\usepackage{verbatim}
\usepackage{amsfonts}
\usepackage{amsmath}
%\usepackage{lineno}
\setlength{\parskip}{2ex} 
%\setlength{\parindent}{0ex}

% Cannot place floats in figure environment.
\usepackage{caption}
\usepackage{newfloat}
%\DeclareCaptionListFormat{myliststyle}{#1.#2}
\DeclareCaptionType{mytype}[Fig.][List of mytype]
\newenvironment{myfigure}{\captionsetup{type=mytype}}{}



\newenvironment{packed_enum}{
\begin{enumerate}
 \setlength{\itemsep}{0pt}
  \setlength{\parskip}{0pt}
  \setlength{\parsep}{0pt}
}{\end{enumerate}}
\newenvironment{packed_item}{
\begin{itemize}
 \setlength{\itemsep}{0pt}
  \setlength{\parskip}{0pt}
  \setlength{\parsep}{0pt}
}{\end{itemize}}

 \newcommand\independent{\protect\mathpalette{\protect\independenT}{\perp}}
    \def\independenT#1#2{\mathrel{\setbox0\hbox{$#1#2$}%
    \copy0\kern-\wd0\mkern4mu\box0}} 


\bibliographystyle{plainnat}

\pagestyle{myheadings}
\markright{Advanced Topics in Causal Inference \hfill  R Lab \#3 \hfill}


\title{R Lab 3 Part 2 - Understanding Time Dependent Confounding and Identifiability in Longitudinal Context}
\author{Advanced Topics in Causal Inference}
\date{}

\begin{document}
\maketitle

\Sconcordance{concordance:RLab3_answers_pt2.tex:RLab3_answers_pt2.Rnw:%
1 303 1}


\noindent \textbf{Assigned:} September 28, 2021\\
\textbf{Lab due:} October 05, 2020 on bCourses. Please answer all questions and include relevant \texttt{R} code. You are encouraged to discuss the assignment in groups, but should not copy code or interpretations verbatim. Upload your own completed lab to bCourses.



\noindent \textbf{Last lab:} \\
Translate causal questions into target causal parameters and intervene on the data generating processes described by Structural Causal Models (SCMs) to evaluate the true value of these parameters. 


\noindent \textbf{This lab:}\\
1. Review the concept of identifiability. \\
2. Determine which assumptions let us achieve identifiability. \\
3. Write the target causal parameter (a function of the counterfactual or ``post intervention" distribution) as a function of the observed data distribution. \\
4. Obtain the value of the statistical estimand. \\
5. Understand the challenges posed by time-dependent confounding.

\noindent \textbf{Next lab:}\\
Implementation of longitudinal IPTW to estimate the intervention specific mean and the parameters of a marginal structural model.



\begin{center}
\noindent\rule{18cm}{0.4pt}
\end{center}

\noindent\large\textbf{Data Structure 2: $O = (L(1), A(1), L(2), A(2), L(3), A(3), L(4), A(4), Y)$}
\normalsize

\begin{enumerate}
\item \underline{SCM}:
\begin{packed_item}
\item[] $U = (U_{L(t)}, U_{A(t)}, U_{Y}), t=1,...,4 \sim P_U$. Assume $U$s are jointly independent.
\item[] Structural equations, $F$:
\begin{align*}
L(1) & = f_{L(1)}(U_{L(1)}) \\
A(1) & = f_{A(1)}(L(1), U_{A(1)}) \\
L(t) & = f_{L(t)}(\bar{L}(t-1), \bar{A}(t-1), U_{L(t)}) \text{ for } t = 2,...,4\\
A(t) & = f_{A(t)}(\bar{A}(t-1), \bar{L}(t), U_{A(t)}) \text{ for } t = 2, ..., 4 \\
Y & = f_Y(\bar{L}(4), \bar{A}(4), U_Y) 
\end{align*}
\end{packed_item}
\textbf{Is the true $P_{U,X}$ compatible with the SCM presented?} Refer back to the second data structure of \texttt{R} Lab 1 for the true $P_{U,X}$.
\item \underline{Target causal parameter}:
\[
\Psi^F(P_{U,X}) = E_{U,X}[Y_{\bar{a}(4)}]
\]
\textbf{Is the target causal parameter (a parameter of $P_{U,X}$) identified (as a parameter of $P_0$) under the standard, point treatment randomization assumption/back door criteria? Why or why not?}
\item \textbf{If the target parameter is not identified in the previous question, what are the alternative assumptions under which the parameter would be identified?}
\item \textbf{What is the corresponding statistical estimand, $\Psi(P_0)$, under these assumptions?}
\item \textbf{Bonus!} Suppose instead your target causal parameter is $E_{U,X}[Y_{a(1)}]$. What is the interpretation of this target parameter? What assumptions are needed for identifiability here? What is the statistical estimand?
\end{enumerate}



\begin{solution}

\begin{enumerate}

\item The $P_{U,X}$ is compatible with the SCM. See the previous data structure for reasoning.

\item The reasoning here is the same as the previous two data structures. We can't condition on one set of covariates to achieve identifiability -- we would either be left with unblocked backdoor paths or a loss of part of the effect of interest. For example, we need to condition on $L(3)$ to block the backdoor path from $A(3)$ to $Y$, but $L(3)$ is a descendant of $A(2)$ so we lose part of the effect of $A(2)$ on $Y$ via $L(3)$.

\item Additional assumptions needed for identifiability:

\begin{itemize}
\item[-] Sequential randomization assumption:

\[
Y_{\bar{a}(t)} \independent A(t) | \bar{L}(t), \bar{A}(t-1) = \bar{a}(t-1) \text{ for } t = 1, ..., 4 
\]

\item[-] Positivity:

\[
P_0(A(t) = a(t) | \bar{A}(t-1) = \bar{a}(t-1), \bar{L}(t)) > 0, \text{ for } t = 1,..., 4
\]

\end{itemize}

\item Under the above assumptions, the target causal parameter is equal to:

$$E_{U,X}[Y_{\bar{a}(4)}] = \sum_{\bar{l}(4)} E_0[Y|\bar{A}(4) = \bar{a}(4), \bar{L}(4) = \bar{l}(4)] \times \prod_{t = 1}^4 P_0(L(t) = l(t)|\bar{A}(t-1) = \bar{a}(t-1), \bar{L}(t-1) = \bar{l}(t-1))$$

\end{enumerate}

\noindent\textbf{Bonus:} If our target parameter is instead $E_{U,X}[Y_{a(1)}]$, this is the expectation of the counterfactual outcome (test score) under an intervention on the first night's sleep only (i.e., one night of either 8 or more hours of sleep or less than 8 hours of sleep at time 1) on the outcome. Subsequent values of $A$ are left random (i.e., they are generated without any further intervention on the data generating system). Contrasting, say $E[Y_{a(1) = 1} - Y_{a(1) = 0}]$ is thus capturing a single point treatment effect of early sleep (in part mediated by future nights' sleep), but is no longer summarizing the cumulative effect of sleep over multiple nights.

Here, we can actually use the standard point treatment randomization assumption -- we only need to condition on one set of covariates (in fact, only one covariate) to achieve the backdoor criteria. That is:

\[
Y_{a(1)} \independent A(1) | L(1)
\]

Positivity assumption:

\[
P_0(A(1) = a(1) | L(1)) > 0 - a.e.
\]

Under these assumptions, the target causal parameter is identified using the standard g-computation formula you learned in Causal I:

\[
E_{U,X}[Y_{a(1)}]=E_{L(1)}[E_0[Y|A(1)=a(1),L(1)]]
\]

\end{solution}

\noindent\large\textbf{Data Structure 4: $O = (L(1), C(1), A(1), Y(2), L(2), C(2), A(2), Y(3))$}
\normalsize


\begin{enumerate}
\item \underline{SCM}:
\begin{packed_item}
\item[] $U = (U_{L(t)}, U_{C(t)}, U_{A(t)}, U_{Y(t+1)}), t=1,2 \sim P_U$. Assume $U$s are jointly independent.
\item[] Structural equations, $F$:
\begin{align*}
L(1) & = f_{L(1)}(U_{L(1)}) \\
C(1) & = f_{C(1)}(L(1), U_{C(1)}) \\
A(1) & = f_{A(1)}(L(1), C(1), U_{A(1)}) \\
Y(2) & = f_{Y(2)}(L(1), C(1), A(1), U_{Y(2)}) \\
L(2) & = f_{L(2)}(L(1), C(1), A(1), Y(2), U_{L(2)}) \\
C(2) & = f_{C(2)}(\bar{L}(2), C(1), A(1), Y(2), U_{C(2)}) \\
A(2) & = f_{A(2)}(\bar{L}(2), \bar{C}(2), A(1), Y(2), U_{A(2)}) \\
Y(3) & = f_{Y(3)}(\bar{L}(2), \bar{C}(2), \bar{A}(2), Y(2), U_{Y(3)}) 
\end{align*}
\end{packed_item}
\textbf{Is the true $P_{U,X}$ an element of the SCM presented?} Refer back to the fourth data structure in \texttt{R} Lab 1 for the true $P_{U,X}$.
\item \underline{Target causal parameter}:
\[
\Psi^F(P_{U,X}) = E_{U,X}[Y(3)_{\bar{a}(2)=1, \bar{c}(2)=0}]
\]
\textbf{Is the target causal parameter (a parameter of $P_{U,X}$) identified (as a parameter of $P_0$) under the standard, point treatment randomization assumption/back door criteria? Why or why not?}
\item \textbf{If the target parameter is not identified in the previous question, what are the alternative assumptions under which the parameter would be identified?}
\item \textbf{What is the corresponding statistical estimand, $\Psi(P_0)$, under these assumptions?}
\end{enumerate}








\begin{solution}
\begin{enumerate}
\item $P_{U,X}$ is contained in the SCM. See data structure 1 for reasoning.
\item The point treatment backdoor criteria is not sufficient here -- there is no one set of covariates that sufficiently blocks backdoor paths from our intervention $\bar{A}(2)$ to $Y$ and are non-descendants of $A(1)$ and $A(2)$.
\item Additional assumptions:
\begin{itemize}
\item[-] Sequential randomization assumption:
\begin{align*}
Y(3)_{\bar{a}=1, \bar{c}=0} &\independent C(t) | \bar{L}(t), \bar{A}(t-1) = 1, \bar{C}(t-1) = 0, Y(t) = 0, \text{ for } t = 1, 2 \\
Y(3)_{\bar{a}=1, \bar{c}=0} &\independent A(t) | \bar{C}(t) = 0, \bar{L}(t), \bar{A}(t-1) = 1, Y(t) = 0, t = 1, 2 
\end{align*}
\item[-] Positivity assumption:
\begin{align*}
P_0(A(t)=1 &| \bar{C}(t)=0, \bar{L}(t), Y(t) = 0, \bar{A}(t-1)=1)>0 \text{ for } t=1,2 \\
P_0(C(t) = 0 &| \bar{L}(t), \bar{A}(t-1)=1, \bar{C}(t-1) =0, Y(t)=0) > 0 \text{ for } t = 1, 2 
\end{align*}
\end{itemize}

Equivalently (since $A(t)$ and $C(t)$ are adjacent nodes), we can write the assumptions as follows:
\begin{itemize}
\item[-] Sequential randomization assumption:
\[
Y(3)_{\bar{a}=1,\bar{c}=0} \independent (A(t),C(t)) | \bar{L}(t),Y(t)=0,C(t-1)=0,\bar{A}(t-1) = 1 \text{ for } t=1,2
\]
\item[-] Positivity assumption:
\[
P_0(A(t) = 1, C(t) = 0 | \bar{L}(t), Y(t)=0, \bar{C}(t-1) = 0, \bar{A}(t-1) = 1) > 0 - a.e. \text{ for } t = 1,2
\]
\end{itemize}



\item Under the above assumptions, $\Psi(P_{U,X}) = \Psi(P_0)$. That is:

\begin{align*}
E_{U,X}[Y(3)_{\bar{a}(2)=1, \bar{c}(2)=0}] = & \sum_{\bar{l}(2), y(2)} \Big{[}P_0(Y(3) = 1 | \bar{A}(2) = 1, \bar{C}(2) = 0, \bar{L}(2) = \bar{l}(2), Y(2) = y(2)) \\
& \times P_0(L(2) = l(2) | Y(2) = y(2), A(1) = 1, C(1) = 0, L(1) = l(1)) \\
& \times P_0(Y(2) = y(2) | A(1) = 1, C(1) = 0, L(1) = l(1)) \\
& \times P_0(L(1) = l(1)) \Big{]}\\
= & \sum_{\bar{l}(2)} \Big{[}\big{[}P_0(Y(3) = 1 | \bar{A}(2) = 1, \bar{C}(2) = 0, \bar{L}(2) = \bar{l}(2), Y(2) = 1) \\
& \times P_0(L(2) = l(2) | Y(2) = 1, A(1) = 1, C(1) = 0, L(1) = l(1)) \\
& \times P_0(Y(2) = 1 | A(1) = 1, C(1) = 0, L(1) = l(1)) \\
& \times P_0(L(1) = l(1))\big{]} \\
& + \big{[}P_0(Y(3) = 1 | \bar{A}(2) = 1, \bar{C}(2) = 0, \bar{L}(2) = \bar{l}(2), Y(2) = 0) \\
& \times P_0(L(2) = l(2) | Y(2) = 0, A(1) = 1, C(1) = 0, L(1) = l(1)) \\
& \times P_0(Y(2) = 0 | A(1) = 1, C(1) = 0, L(1) = l(1)) \\
& \times P_0(L(1) = l(1))\big{]}\Big{]} \\
= & \sum_{\bar{l}(1)} P_0(Y(2) = 1 | A(1) = 1, C(1) = 0, L(1) = l(1)) \\
& \times P_0(L(1) = l(1)) ]\\
& + \sum_{\bar{l}(2)} [P_0(Y(3) = 1 | \bar{A}(2) = 1, \bar{C}(2) = 0, \bar{L}(2) = \bar{l}(2), Y(2) = 0) \\
& \times P_0(L(2) = l(2) | Y(2) = 0, A(1) = 1, C(1) = 0, L(1) = l(1)) \\
& \times P_0(Y(2) = 0 | A(1) = 1, C(1) = 0, L(1) = l(1)) \\
& \times P_0(L(1) = l(1))]
\end{align*}

To understand the third equality, note that if $Y(2) = 1$ (e.g., the student becomes ill at time 1), $Y(3)$ and $L(2)$ will deterministically be equal to some value.

For illustration, consider the case where $L(2)$ is defined as deterministically taking value 0 after the event occurs. You can work out for yourself that this equality also holds under different ways of deterministically assigning a value to $L(2)$ after the event, such as equal to its last observed value. 

Thus, $P_0(Y(3) = 1 | \bar{A}(2) = 1, \bar{C}(2) = 0, \bar{L}(2) = \bar{l}(2), Y(2) = 1) = 1$ and for a given value of $l(1)$, $P_0(L(2) = l(2) | Y(2) = 1, A(1) = 1, C(1) = 0, L(1) = l(1))$ will be equal to 1 for $l(2) = 0$ and equal to 0 for all possible values of $l(2)$. Note that it doesn't matter what value we set $L(2)$ equal to deterministically when $Y(2) = 1$; however, when we set it, the g-computation formula reduces to the above formula. 

\end{enumerate}
\end{solution}

\pagebreak
\section{For Your Project: Identification}

Think through the following questions and apply them to the dataset you will use for your final project.

\begin{enumerate}
\item Under what assumptions is the target causal parameter you came up with in the previous lab identified as a function of the observed data distribution?
\item What is your $\Psi(P_0)$, the statistical estimand?
\item Optional: confirm that in your simulation, the value of your estimand equals the value of your target causal parameter.
\end{enumerate}

\pagebreak

\section{Feedback}

Please attach responses to these questions to your lab. Thank you in advance!

\begin{enumerate}
\item Did you catch any errors in this lab? If so, where?
\item What did you learn in this lab?
\item Do you think that this lab met the goals listed at the beginning? 
\item What else would you have liked to review? What would have helped your understanding?
\item Any other feedback?
\end{enumerate}



\end{document}


