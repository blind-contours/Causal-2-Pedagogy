% if you want the answers to appear uncomment the below
\documentclass[answers]{exam}
% otherwise uncomment the below
% \documentclass{exam}

\usepackage{graphicx}
\usepackage[letterpaper, margin=.9in]{geometry}
\usepackage{natbib}
\bibpunct{(}{)}{;}{a}{,}{,}
\usepackage{url}
\def\UrlFont{\rm}

\usepackage{Sweave}

\usepackage[utf8x]{inputenc}
\usepackage{array}
\usepackage{verbatim}
\usepackage{amsfonts}
\usepackage{amsmath}
%\usepackage{lineno}
\setlength{\parskip}{2ex} 
%\setlength{\parindent}{0ex}

% Cannot place floats in figure environment.
\usepackage{caption}
\usepackage{newfloat}
%\DeclareCaptionListFormat{myliststyle}{#1.#2}
\DeclareCaptionType{mytype}[Solution Fig.][List of mytype]
\newenvironment{myfigure}{\captionsetup{type=mytype}}{}



\newenvironment{packed_enum}{
\begin{enumerate}
 \setlength{\itemsep}{0pt}
  \setlength{\parskip}{0pt}
  \setlength{\parsep}{0pt}
}{\end{enumerate}}
\newenvironment{packed_item}{
\begin{itemize}
 \setlength{\itemsep}{0pt}
  \setlength{\parskip}{0pt}
  \setlength{\parsep}{0pt}
}{\end{itemize}}

 \newcommand\independent{\protect\mathpalette{\protect\independenT}{\perp}}
    \def\independenT#1#2{\mathrel{\setbox0\hbox{$#1#2$}%
    \copy0\kern-\wd0\mkern4mu\box0}} 


\bibliographystyle{plainnat}

\pagestyle{myheadings}
\markright{Advanced Topics in Causal Inference \hfill  R Lab \#6 \hfill}


\title{R Lab 6 - Estimation, Part II: parametric g-computation, ICE representation of g-formula, and TMLE for intervention specific mean}
\author{Advanced Topics in Causal Inference}
\date{}


\begin{document}
\maketitle

\Sconcordance{concordance:RLab6_answers.tex:RLab6_answers.Rnw:%
1 184 1 1 4 13 1 1 2 4 0 1 2 3 1 1 6 8 0 1 2 4 1 1 3 5 0 1 4 5 0 1 8 9 %
0 1 2 10 1 1 2 4 0 1 2 6 1 1 2 4 0 2 2 4 0 1 2 2 1 1 5 7 0 2 2 4 0 1 2 %
14 1 1 41 43 0 1 3 1 0 1 1 6 0 1 3 7 1 1 2 1 0 2 1 3 0 1 2 2 1 1 2 4 0 %
2 2 4 0 1 2 11 1 1 26 28 0 1 3 1 0 1 1 6 0 1 2 15 1 1 2 4 0 1 2 1 1 1 2 %
4 0 1 2 2 1 1 2 4 0 1 2 5 1 1 2 4 0 2 2 4 0 1 2 3 1 1 2 4 0 1 2 1 3 5 0 %
2 2 4 0 1 2 21 1 1 53 55 0 1 3 1 0 1 1 6 0 1 2 23 1 1 3 5 0 1 4 2 0 1 1 %
3 0 1 12 13 0 1 4 8 0 1 3 7 0 1 3 8 0 1 2 5 1 1 4 18 1 1 5 7 0 1 2 4 1 %
1 3 5 0 1 7 8 0 1 2 21 1 1 4 6 0 2 2 4 0 1 2 15 1 1 62 64 0 2 2 1 0 1 1 %
6 0 1 2 9 1 1 2 1 0 4 1 3 0 1 2 16 1 1 46 48 0 1 3 1 0 1 1 6 0 1 2 25 1 %
1 2 4 0 1 2 2 1 1 2 4 0 1 2 6 1 1 3 5 0 2 2 4 0 1 2 3 1 1 2 4 0 1 2 1 3 %
5 0 2 2 4 0 1 2 6 1 1 3 5 0 2 2 4 0 1 2 3 1 1 2 4 0 2 2 4 0 2 2 4 0 1 2 %
16 1 1 2 4 0 1 2 5 1 1 82 85 0 1 4 1 0 1 1 6 0 1 2 18 1 1 3 5 0 1 4 2 0 %
1 1 3 0 1 11 9 0 1 3 7 0 1 3 7 0 1 4 12 0 1 3 30 1}


\maketitle
\noindent \textbf{Assigned:} November 2, 2021\\
\textbf{Lab due:} November 10, 2021 on bCourses. Please answer all questions and include relevant \texttt{R} code. You are encouraged to discuss the assignment in groups, but should not copy code or interpretations verbatim. Upload your own completed lab to bCourses.



\noindent \textbf{Last lab:} \\ 
Introduction to \texttt{ltmle} package.\\


\noindent \textbf{Goals for this lab:} \\
1. Implement the ``traditional" longitudinal parametric g-computation estimator \\
2. Implement the g-computation estimator based on the ICE representation of the longitudinal g-computation formula. \\
3. Implement TMLE based on the ICE representation of the longitudinal g-computation formula (by hand). \\


\noindent \textbf{Next lab:}\\
TMLE for intervention specific mean, MSM, and dynamic regimes using the \texttt{ltmle} package.

\begin{center}
\noindent\rule{18cm}{0.4pt}
\end{center}

\section{Introduction and Motivation}



Recall that in \texttt{R} Lab 3, we showed that under certain sequential randomization and positivity assumptions, we can write our target causal parameter, $\Psi(P_{U,X})$, as a parameter of our observed data distribution, $\Psi(P_0)$. Specifically:

\[
\underbrace{E_{U,X}[Y_{\bar{a}}]}_{\Psi(P_{U,X})} \overbrace{=}^{\text{SRA and pos.}} \underbrace{\sum_{\bar{l}}\Big{[}E_0[Y|\bar{A} = \bar{a}, \bar{L} = \bar{l}]\times \prod_{t=1}^KP_0[L(t) | \bar{A}(t-1) = \bar{a}(t-1), \bar{L}(t-1) = \bar{l}(t-1)] \Big{]}}_{\Psi(P_0)}
\]

\noindent $\Psi(P_0)$ in this equation is called the longitudinal g-computation formula, and in this lab we will estimate this statistical parameter using the following three methods:

\begin{enumerate}
\item \textbf{\underline{Parametric g-computation estimator}} - going forwards in time, we estimate the conditional distribution/density of non-intervention nodes, given the past, and plug those estimates into the parameter mapping $\Psi(P_0)$.
\begin{enumerate}
\item Estimate $\bar{Q}(L(t)|\bar{L}(t-1), \bar{A}(t-1))$ the conditional distribution/density of each $L(t)$ given its past parents, for $t = 1,...,K$
\item Use these estimates of $\bar{Q}(L(t)|\bar{L}(t-1), \bar{A}(t-1))$ to ``simulate counterfactual covariate histories'' over time, setting $A(t) = a(t)$ for $t = 1,...,K$
\item Repeat this many times for each treatment of interest
\item Summarize these regime-specific estimates according to the parameter of interest
\end{enumerate}
\item \textbf{\underline{ICE}} - this estimator is the g-computation formula represented as a series of iterated conditional expectations (ICEs). Essentially, we fit a series of regressions going backwards in time, where each regression uses the regression before it (evaluated at the treatment history of interest) as its outcome. 
\begin{enumerate}
\item At $t = K+1$: estimate $\bar{Q}_{K+1}(\bar{a}(K), \bar{L}(K))$
\begin{itemize}
\item[-] In other words, $E[Y|\bar{L}(K), \bar{A}(K) = \bar{a}(K)]$
\end{itemize}
\item At $t = K$: estimate $\bar{Q}_{K}(\bar{a}(K-1), \bar{L}(K-1))$ by using the estimated $\bar{Q}_{K+1}(\bar{a}(K), \bar{L}(K))$ as an outcome and regressing it on $\bar{L}(K-1), \bar{A}(K-1) = \bar{a}(K-1)$
\begin{itemize}
\item[-] In other words, take the expectation of $E[Y|\bar{L}(K), \bar{A}(K) = \bar{a}(K)]$ with respect to the distribution of $L(K)$ given $\bar{L}(K-1)$ and $\bar{A}(K-1) = \bar{a}(K-1)$
\item[-] Also denoted $E\big{[}E[Y|\bar{L}(K), \bar{A}(K) = \bar{a}(K)] | \bar{L}(K-1), \bar{A}(K-1) = \bar{a}(K-1)\big{]}$
\end{itemize}
\item $\ldots$
\item At $t = 2$: estimate $\bar{Q}_2(a(1), L(1))$ by using $\bar{Q}_{3}(\bar{a}(2), \bar{L}(2))$ as an outcome and regressing it on $L(1), A(1) = a(1)$
\begin{itemize}
\item[-] In other words, take the expectation of $E\Big{[}\ldots\big{[}E[E[Y|\bar{L}(K), \bar{A}(K) = \bar{a}(K)] | \bar{L}(K-1), \bar{A}(K-1) = \bar{a}(K-1)]\big{]}\ldots\Big{]}$ with respect to the distribution of $L(2)$ given $L(1)$ and $A(1) = a(1)$ 
\end{itemize}
\item Plug in the estimate of $\bar{Q}_2(a(1), L(1))$ into the parameter mapping $\Psi(P_0)$, i.e., take the empirical mean of the estimated $\bar{Q}_2(a(1), L(1))$
\end{enumerate}
\item \textbf{\underline{TMLE}} - similar to the ICE g-computation estimator, except that we update each regression at time $t$ before using it as an outcome for the next regression corresponding to time $t-1$.
\begin{enumerate}
\item At $t = K+1$
\begin{enumerate}
\item Estimate $\bar{Q}^0_{K+1}(\bar{a}(K), \bar{L}(K))$.
\item Target initial estimate of $\bar{Q}^0_{K+1}(\bar{a}(K), \bar{L}(K))$ and update to $\bar{Q}^\star_{K+1}(\bar{a}(K), \bar{L}(K))$  by ``cleverly" incorporating an estimate of the treatment mechanism at $K$, $\prod_{j=1}^Kg(A(j)|\bar{A}(j-1),\bar{L}(j))$
\end{enumerate}
\item At $t = K$
\begin{enumerate}
\item Using $\bar{Q}^\star_{K+1}(\bar{a}(K), \bar{L}(K))$ as an outcome, estimate $Q^0_K(\bar{a}(K-1), \bar{L}(K-1))$. 
\item Target $\bar{Q}_n,K(\bar{a}(K-1), \bar{L}(K-1))$ and update to $\bar{Q}^\star_K(\bar{a}(K-1), \bar{L}(K-1))$ by ``cleverly" incorporating an estimate of the treatment mechanism at $K-1$, $\prod_{j=1}^{K-1}g(A(j)|\bar{A}(j-1),\bar{L}(j))$
\end{enumerate}
\item \ldots
\item At $t = 2$
\begin{enumerate}
\item Using $\bar{Q}^\star_3(\bar{a}(2), \bar{L}(2))$ as an outcome, estimate $\bar{Q}^0_2(a(1), L(1))$.
\item Target $\bar{Q}^0_2(a(1), L(1))$ and update to $Q^\star_2(a(1), L(1))$ by ``cleverly" incorporating the treatment mechanism at time 1, $g(A(1)|L(1))$
\end{enumerate}
\item Plug in the targeted estimate of $\bar{Q}^\star_2(a(1), L(1))$ into the parameter mapping $\Psi(P_0)$, i.e., take the empirical mean of $\bar{Q}^\star_2(a(1), L(1))$

\end{enumerate}

\end{enumerate}

\begin{figure}
\begin{center}
\includegraphics[width=.4\textwidth]{Gcomp.png}
\caption{G-computation.}
\end{center}
\end{figure}

\subsection{This lab}

Recall that your GSR gave you data on one-thousand students, which we can treat as 1,000 i.i.d. copies of $O$. In \texttt{R} Lab 4 we used these data to estimate the effects of sleep using the IPTW estimator, and measured the estimator's performance. 

In this lab we are continuing estimation (step 6 of the roadmap) of the effects of sleep by implementing the 3 estimators above. You'll also evaluate the estimators' performance (refer to \texttt{R} Lab 4 for definitions). In this lab, again you will answer questions for two of the data generating systems we've been working with in previous labs. 

\subsection{To turn in:}


\noindent\fbox{
    \parbox{\textwidth}{

\textbf{\underline{For each of the 2 data structures listed below, answer the following questions:}} \\



\begin{enumerate}
\item \textbf{Implement: (1) traditional longitudinal parametric g-computation estimator, (2) g-computation estimator based on the ICE representation of the longitudinal computation formula, (3) TMLE based on ICE representation of longitudinal g-computation}. 
\item \textbf{Compare performance metrics of each of the estimators}. Specifically, evaluate the bias, variance, and mean-squared error (MSE) of each of the estimators.
\end{enumerate}
    
    }
}

\pagebreak
\noindent\large\textbf{Data Structure 0: $O = (L(1), A(1), L(2), A(2), Y)$}
\normalsize
We are interested in the expected $Y$ if everyone got treatment regime $\bar{a}(2) = 1$.
\[
\Psi^F(P_{U,X}) = E_{U,X}[Y_{\bar{a}(2)=1}] = P_{U,X}(Y_{\bar{a}(2)=1} = 1) = 0.7921
\]

\noindent Before we get started with estimation...
\begin{enumerate}
\item Load \texttt{DataStructure0.RData} using the \texttt{load()} function. Make sure you have specified the correct file path. You should see 4 new things come up in your global environment:

\begin{packed_item}
\item[-] \texttt{ObsData0} - this is a dataframe of 1,000 observations that follows Data Structure 0 from the previous lab.
\item[-] \texttt{Psi.F0} - this is the true $\Psi^F(P_{U,X})$ value for the target causal parameter $E_{U,X}[Y_{\bar{a}(2)=1}]$ (generated in lab 3).
\item[-] \texttt{generate\_data0} - this is the function that generates $n$ copies of Data Structure 0.
\item[-] \texttt{generate\_data0\_intervene} - this is the function that generates $n$ intervened-on copies of Data Structure 0. We won't be using this function in this lab, so you can remove it using the \texttt{rm()} function.
\begin{Schunk}
\begin{Sinput}
> rm(generate_data0_intervene)
\end{Sinput}
\end{Schunk}
\end{packed_item}

\item Assign the number of students to \texttt{n}.
\item Create a new function, \texttt{bound()}, that takes as input a number \texttt{x} and bounds it to be between 0.1 and 1. That is: \texttt{x} is greater than 1, return 1; when \texttt{x} is smaller than 0.01, return .01.
\begin{Schunk}
\begin{Sinput}
> bound = function(x){
+   x[x < 0.01] = 0.01
+   x[x > 1] = 0.99
+   return(x)
+ }
\end{Sinput}
\end{Schunk}
This function will be useful to prevent extreme weights on our clever covariate in the TMLE sections.
\item Also make sure you have the \texttt{bound()} function (defined in the previous data structure) ready and living in your global environment.
\end{enumerate}

\begin{solution}
\begin{Schunk}
\begin{Sinput}
> # load ObsData0, Psi.F0, generate_data0
> load("DataStructure0.RData")
\end{Sinput}
\end{Schunk}
\begin{Schunk}
\begin{Sinput}
> # set number of observations to the number of rows
> n = nrow(ObsData0)
\end{Sinput}
\end{Schunk}
\begin{Schunk}
\begin{Sinput}
> # function that bounds x between 0.01 and 1
> bound = function(x){
+   x[x < 0.01] = 0.01
+   x[x > 1] = 0.99
+   return(x)
+ }
\end{Sinput}
\end{Schunk}
\end{solution}


%%%%% PARAMETRIC GCOMP OBSDATA0 %%%%%%%%
\noindent \textbf{\underline{Parametric g-computation}} - Data Structure 0
\begin{enumerate}
\item Create a new function \texttt{param.gcomp0\_fun()} that takes in \texttt{abar} as an argument. Within the function:
\begin{enumerate}
\item First, estimate the conditional distributions of all the non-intervention nodes, in this case, $L(1)$, $L(2)$, and $Y$. For this example, we need estimates of:
\begin{enumerate}
\item $\bar{Q}(Y|\bar{L}(2), \bar{A}(2))$, or the conditional probability of $Y$, given the past. For example:
\begin{Schunk}
\begin{Sinput}
> Q.Y.reg = glm(Y ~ L1 + A1 + L2 + A2, data = ObsData0, family = "binomial")
\end{Sinput}
\end{Schunk}
\item $\bar{Q}(L(2)|L(1), A(1))$, or the conditional probability of $L(2)$, given the past.
\item $\bar{Q}(L(1))$, the baseline covariate distribution. To estimate the distribution of $L(1)$, we can use the empirical distribution. \emph{Hint:} go to the next step!
\end{enumerate}
\item Use these estimates to generate (using Monte Carlo simulation) many ``counterfactual" covariate and outcome histories over time, setting $A(t) = a(t)$ for $t = 1,2$.  
\begin{enumerate}
\item Set \texttt{S} equal to $10,000$, the number of times we will simulate.
\item Sample \texttt{S} observations/rows, with replacement, from \texttt{ObsData0}.
\begin{Schunk}
\begin{Sinput}
> ObsData0.MC = ObsData0[sample(1:n, S, replace = T),]
\end{Sinput}
\end{Schunk}
\item Draw $L_i(1) = l_i(1)$ for each individual. That is, set \texttt{l1} equal to \texttt{L1} that lives within the simulated data:
\begin{Schunk}
\begin{Sinput}
> l1 = ObsData0.MC$L1
\end{Sinput}
\end{Schunk}
\item Draw $L_i(2) = l_i(2)$ for each individual. 
\begin{enumerate}
\item From the estimated conditional distribution of $Q_n(L(2)| L(1), A(1))$, predict the probability of $L(2) = 1$, setting baseline exposure and covariate to $a(1)$ and $l(1)$, respectively. That is:
\begin{Schunk}
\begin{Sinput}
> Q.L2 = predict(Q.L2.reg, 
+                 type = "response",
+                 newdata = data.frame(L1 = l1,
+                                      A1 = abar[1]))
\end{Sinput}
\end{Schunk}
\item Because $L(2)$ is binary, for each individual, draw an observation from a $Bernoulli(p_i = \bar{Q}(L_i(2)| L_i(1) = l_i(1), A_i(1) = a_i(1)))$ distribution.
\begin{Schunk}
\begin{Sinput}
> l2 = rbinom(S, size = 1, prob = Q.L2)
\end{Sinput}
\end{Schunk}
\end{enumerate}
\item Draw $Y_i = y_i$ for each individual. 
\begin{enumerate}
\item From the estimated conditional distribution of $Q_n(Y| \bar{L}(2), \bar{A}(2))$, predict the probability of $Y = 1$, setting time varying exposures and covariates to $\bar{a}(2)$ and $\bar{l}(2)$, respectively.
\item Because $Y$ is binary, for each individual, draw an observation from a $Bernoulli(p_i = Q_n(Y_i | L_i(1) = l_i(1), A_i(1) = a_i(1), L_i(2) = l_i(2), A_i(2) = a_i(2)))$ distribution.
\end{enumerate}
\end{enumerate}
\item Estimate the probability of $Y=1$ with the empirical proportion of \texttt{y}. Have the function \texttt{param.gcomp0\_fun()} return this value.
\end{enumerate}
\item Estimate $E_{U,X}[Y_{\bar{a} = 1}] = P_{U,X}(Y_{\bar{a} = 1} = 1)$ using parametric g-computation by applying \texttt{param.gcomp0\_fun()}, with the correct \texttt{abar} argument specification.
\end{enumerate}



\begin{solution}
\begin{Schunk}
\begin{Sinput}
> ### Estimation of parametric g-computation - ObsData0 ###
> # create function that returns estimate of param. g-comp
> param.gcomp0_fun = function(abar){
+   
+   # Estimate conditional dist of non-intervention nodes #
+   # estimate Q(L(2)|L(1), A(1))
+   Q.L2.reg = glm(L2 ~ L1 + A1, data = ObsData0, family = "binomial")
+   # estimate Q(Y|Lbar(2), Abar(2))
+   Q.Y.reg = glm(Y ~ L1 + A1 + L2 + A2, data = ObsData0, family = "binomial")
+   
+   # number of times to MC simulate
+   S = 10000
+   
+   # sample rows/observations with replacement, S times
+   ObsData0.MC = ObsData0[sample(1:n, S, replace = T),] 
+ 
+   # draw L1 for each individual
+   l1 = ObsData0.MC$L1
+   
+   # draw L2 for each individual
+   Q.L2 = predict(Q.L2.reg, 
+                  type = "response",
+                  newdata = data.frame(L1 = l1, 
+                                       A1 = abar[1]))
+   l2 = rbinom(S, size = 1, prob = Q.L2)
+   
+   # draw Y for each individual
+   Q.Y = predict(Q.Y.reg, type = "response",
+                 newdata = data.frame(L1 = l1, 
+                                      L2 = l2,
+                                      A1 = abar[1], 
+                                      A2 = abar[2]))
+   y = rbinom(S, size = 1, prob = Q.Y)
+   
+   # proportion of simulated Y = 1 is estimate
+   propY = mean(y)
+   
+   return(propY)
+ 
+ }
\end{Sinput}
\end{Schunk}
\begin{Schunk}
\begin{Sinput}
> Psi.hat.pgcomp0 = param.gcomp0_fun(abar = c(1,1))
> Psi.hat.pgcomp0
\end{Sinput}
\begin{Soutput}
[1] 0.789
\end{Soutput}
\end{Schunk}
\end{solution}

%%%%% ICE OBSDATA0 %%%%%%%%
\noindent\textbf{\underline{ICE g-computation}} - Data Structure 0
\begin{enumerate}
\item Create a new function \texttt{ICE.gcomp0\_fun()} that takes in \texttt{abar} as an argument. Within the function:
\begin{enumerate}
\item Create a dataframe called \texttt{newdata} where $A(1)$ and $A(2)$ have been set to $a(1)$ and $a(2)$, respectively:
\begin{Schunk}
\begin{Sinput}
> newdata = ObsData0
> newdata$A1 = abar[1]
> newdata$A2 = abar[2]
\end{Sinput}
\end{Schunk}
\item For time $t = 3$:
\begin{enumerate}
\item Regress $Y$ on observed history at time 3 using \texttt{glm()}.
\begin{Schunk}
\begin{Sinput}
> Q3.reg = glm(Y ~ L1 + A1 + L2 + A2, family = "binomial", data = ObsData0)
\end{Sinput}
\end{Schunk}
\item Generate predicted values of $Y$, evaluated at each observed covariate value and exposure history of interest (i.e., $\bar{L}_i(2) = \bar{l}_i(2)$ and $\bar{A}_i(2) = \bar{a}(2)$, respectively). Set the predicted values equal to \texttt{Q3}.
\begin{Schunk}
\begin{Sinput}
> Q3 = predict(Q3.reg, type = "response", newdata = newdata)
\end{Sinput}
\end{Schunk}
\end{enumerate}
\item For time $t = 2$:
\begin{enumerate}
\item Using the predicted values from the prior step as a new outcome (i.e., \texttt{Q3}), regress the new outcome on the past at time $2$.
\item Generate new predicted values, evaluated at each observed covariate value and exposure history of interest (i.e., $L_i(1) = l_i(1)$ and $A_i(1) = a(1)$, respectively)
\end{enumerate}
\item Take the empirical mean of the final predicted outcome. This is the value that \texttt{ICE.gcomp0\_fun()} should return.
\end{enumerate}
\item Estimate $E_{U,X}[Y_{\bar{a} = 1}] = P_{U,X}(Y_{\bar{a} = 1} = 1)$ using parametric g-computation by applying \texttt{ICE.gcomp0\_fun()}, with the correct \texttt{abar} argument specification.
\end{enumerate}

\begin{solution}
\begin{Schunk}
\begin{Sinput}
> ### Estimation of ICE g-computation - ObsData0 ###
> ICE.gcomp0_fun = function(abar){
+   
+   newdata = ObsData0
+   newdata$A1 = abar[1]
+   newdata$A2 = abar[2]
+   
+   # regress Y on observed past
+   Q3.reg = glm(Y ~ ., family = "binomial", data = ObsData0)
+   
+   # generate predicted values of Y at exposure history we want = Q3
+   Q3 = predict(Q3.reg, type = "response", newdata = newdata)
+   
+   # regress Ypred1 on observed past
+   Q2.reg = glm(Q3 ~ A1 + L1, data = ObsData0, family = "quasibinomial")
+   
+   # generate predicted values at exposure history we want = Q2
+   Q2 = predict(Q2.reg, type = "response", newdata = newdata)
+   
+   # mean of Ypred2 is estimate
+   meanQ2 = mean(Q2)
+   
+   return(meanQ2)
+   
+ }
\end{Sinput}
\end{Schunk}
\begin{Schunk}
\begin{Sinput}
> Psi.hat.ICE0 = ICE.gcomp0_fun(abar = c(1,1))
> Psi.hat.ICE0
\end{Sinput}
\begin{Soutput}
[1] 0.7798641
\end{Soutput}
\end{Schunk}
\end{solution}

%%%%% TMLE OBSDATA0 %%%%%%%%

\noindent \textbf{\underline{TMLE}} - Data Structure 0
\begin{enumerate}
\item Create a new function \texttt{TMLE.gcomp0\_fun()} that takes in \texttt{abar} as an argument. Within the function:
\begin{enumerate}
\item Create a dataframe called \texttt{newdata} where $A(1)$ and $A(2)$ have been set to $a(1)$ and $a(2)$, respectively.
\item Estimate the probability of receiving treatment $P_0(A(t)=1|\bar{L}(t), \bar{A}(t-1)) = g_0(A(t)=1|\bar{L}(t), \bar{A}(t-1))$ for $t = 1,2$ using correctly specified parametric regression models. The correct model specifications are (refer back to \texttt{R} lab 3 for reference):
\begin{align*}
g_0(A(1)=1|L(1)) & = expit[\beta_0 + \beta_1L(1)] \\
g_0(A(2)=1|\bar{L}(2)) & = expit[\beta_0 + \beta_1L(1) + \beta_2L(2)]
\end{align*}
\begin{enumerate}
\item Use the \texttt{glm()} function, and specify the correct formula following the above model specifications. Also include the arguments \texttt{family = 'binomial'} for logistic regression and \texttt{data = ObsData0}. For example, for $t = 1$:
\begin{Schunk}
\begin{Sinput}
> gA1.reg = glm(A1 ~ L1, family = 'binomial', data = ObsData0)
\end{Sinput}
\end{Schunk}
Do the same for $t = 2$.
\item Predict each subject's probability of receiving treatment at time $t$, given his or her observed treatment and covariate history i.e., $g_n(A_i(t)=1|\bar{A}_i(t-1), \bar{L}_i(t))$. Assign to the variables \texttt{g.abar1.1} and \texttt{g.abar2.1} (respectively). For example, for $t = 2$:
\begin{Schunk}
\begin{Sinput}
> g.abar2.1 = predict(gA2.reg, type = "response")
\end{Sinput}
\end{Schunk}
Do the same for $t = 1$.
\end{enumerate}
\item For each timepoint $t = 1, 2$, create a logical variable that indicates which students had a treatment history $\bar{A}(t) = \bar{a}(t)$. For example for $t = 1$:
\begin{Schunk}
\begin{Sinput}
> I1 = ObsData0$A1 == abar[1] 
\end{Sinput}
\end{Schunk}
Do the same for $t = 2$.
\item For $t = 3$:
\begin{enumerate}
\item Estimate $\bar{Q}^0_{3,n}(\bar{a}(2), \bar{L}(2))$. This is the conditional probability of $Y = 1$ (i.e., \texttt{Q3}), given the past covariates and exposure of interest.
\begin{enumerate}
\item Regress $Y$ on the \textit{observed past}:
\begin{Schunk}
\begin{Sinput}
> Q3.reg = glm(Y ~ L1 + A1 + L2 + A2, data = ObsData0, family = "binomial")
\end{Sinput}
\end{Schunk}
\item Using the above model, predict $\bar{Q}^0_{3,n}$, the conditional probability on the logit scale for all subjects \textit{at the exposure history we want} (i.e., $\bar{A}(2) = \bar{a}(2)$). Call this vector \texttt{logit.Q3}.
\begin{Schunk}
\begin{Sinput}
> logit.Q3 = predict(Q3.reg, type = "link", newdata = newdata)
\end{Sinput}
\end{Schunk}
\end{enumerate}
\item Update $\bar{Q}^0_{3,n}(\bar{a}(2), \bar{L}(2))$ to $\bar{Q}^\star_{3,n}(\bar{a}(2), \bar{L}(2))$. 
\begin{enumerate}
\item Make the clever covariate, $H_{n,3}(\bar{A}(2), \bar{L}(2)) = \frac{\mathbb{I}[\bar{A}(2) = \bar{a}(2)]}{\prod_{t=1}^2g_n(A(t)|\bar{L}(t), \bar{A}(t-1))}$, where the denominator, $\prod_{t=1}^2g_n(A(t)|\bar{L}(t), \bar{A}(t-1))$, is bounded between 0.01 and 1:
\begin{Schunk}
\begin{Sinput}
> H3 = I2/bound(g.abar1.1 * g.abar2.1)
\end{Sinput}
\end{Schunk}
\item Fit a logistic regression of $Y$ on the intercept, with $logit(\bar{Q}^0_{3,n})$ as an offset and clever covariate $H_{n,3}(\bar{A}(2), \bar{L}(2))$ as weights:
\begin{Schunk}
\begin{Sinput}
> Q3.reg.update = glm(Y ~ offset(logit.Q3), weights = H3, family = "binomial", 
+                     data = ObsData0)
\end{Sinput}
\end{Schunk}
\item Generate $\bar{Q}^\star_{3,n}$, the updated predicted probabilities of $\bar{Q}^0_{3,n}$ using the updated model in the previous step:
\begin{Schunk}
\begin{Sinput}
> Q3.star = predict(Q3.reg.update, type = "response")
\end{Sinput}
\end{Schunk}
\end{enumerate}
\end{enumerate}
\item For $t = 2$:
\begin{enumerate}
\item Estimate $\bar{Q}^0_{2,n}(a(1), L(1))$. This is the conditional expectation of $\bar{Q}^\star_{3,n}$, given the past covariates and exposure of interest.
\begin{enumerate}
\item Regress $\bar{Q}^\star_{3,n}$ on the \textit{observed past}. Call this \texttt{Q2.reg}. Make sure to specify \texttt{data = ObsData0} and \texttt{family = 'quasibinomial'}.
\item Using the above model, predict $logit(\bar{Q}_{2,n})$, the conditional probabiity on the logit scale for all subjects at the exposure history we want (i.e., $A(1) = a(1)$). Call this vector \texttt{logit.Q2}.
\end{enumerate}
\item Update $\bar{Q}^0_{2,n}(a(1), L(1))$ to $\bar{Q}^\star_{2,n}(a(1), L(1))$
\begin{enumerate}
\item Make the clever covariate, $H_{n,2}(A(1), L(1)) = \frac{\mathbb{I}[A(1) = a(1)]}{g_n(A(1)|L(1))}$, where $g_n(A(1)|L(1))$ is bounded between 0.01 and 1. Call this \texttt{H2}.
\item Fit a logistic regression on the intercept, with $logit(\bar{Q}^0_{1,n})$ as an offset and clever covariate $H_{n,2}(A(1), L(1))$ as weights.
\item Generate $\bar{Q}^\star_{2,n}$, the updated predicted probabilities of $\bar{Q}_{2,n}$ using the updated model in the previous step.
\end{enumerate}
\end{enumerate}
\item Take the mean of $\bar{Q}^\star_{2,n}$. This is your TMLE estimate! Have \texttt{TMLE.gcomp0\_fun()} return this value.
\end{enumerate}
\item Estimate $E_{U,X}[Y_{\bar{a} = 1}] = P_{U,X}(Y_{\bar{a} = 1} = 1)$ using TMLE by applying \texttt{TMLE.gcomp0\_fun()}, with the correct \texttt{abar} argument specification.
\end{enumerate}

\begin{solution}
\begin{Schunk}
\begin{Sinput}
> ### Estimation of TMLE - ObsData0 ###
> TMLE.gcomp0_fun = function(abar){
+   
+   # newdata at treatment we want
+   newdata = ObsData0
+   newdata$A1 = abar[1]
+   newdata$A2 = abar[2]
+   
+   # estimate treatment mechanism at each point
+   gA1.reg = glm(A1 ~ L1, family = "binomial", data = ObsData0)
+   gA2.reg = glm(A2 ~ L1 + L2, family = "binomial", data = ObsData0)
+   
+   # with above model, compute predicted probability that A(t) = 1, given observed history
+   g.abar1.1 = predict(gA1.reg, type = "response")
+   g.abar2.1 = predict(gA2.reg, type = "response")
+   
+   # indicator of each abar(t) = 1 for t = 1,2
+   I2 = ObsData0$A1 == abar[1] & ObsData0$A2 == abar[2]
+   I1 = ObsData0$A1 == abar[1]
+ 
+  
+   ###### t = 3 ######
+   # fit model for E[Q3|Lbar(2), Abar(2)] 
+   Q3.reg = glm(Y ~ L1 + A1 + L2 + A2, data = ObsData0, family = "binomial") 
+   
+   # predict conditional prob of outcome at exposure history we want on logit scale
+   logit.Q3 = predict(Q3.reg, type = "link", newdata = newdata)
+   
+   # make clever covariate
+   H3 = I2/ bound(g.abar1.1 * g.abar2.1)
+   
+   # update E[Q3|Lbar(2), Abar(2)] by running logistic regression
+   # with logit(Q3) as offset and clever covariate H3 as weights
+   Q3.reg.update = glm(Y ~ offset(logit.Q3), weights = H3, family = "binomial", data = ObsData0)
+   
+   # update the predicted probabilities of Q3 using above model
+   Q3.star = predict(Q3.reg.update, type = "response")
+ 
+   
+   ###### t = 2 ######
+   Q2.reg = glm(Q3.star ~ L1 + A1, data = ObsData0, family = "quasibinomial") 
+   logit.Q2 = predict(Q2.reg, type = "link", newdata = newdata) 
+   H2 = I1 / bound(g.abar1.1)
+   Q2.reg.update = glm(Q3.star ~ offset(logit.Q2), weights = H2, family = "quasibinomial")
+   Q2.star = predict(Q2.reg.update, type = "response")
+   
+   # average of Q2.star is estimate
+   meanQ.star = mean(Q2.star)
+   
+   return(meanQ.star)
+   
+ }
\end{Sinput}
\end{Schunk}
\begin{Schunk}
\begin{Sinput}
> Psi.hat.TMLE0 = TMLE.gcomp0_fun(abar = c(1,1))
> Psi.hat.TMLE0
\end{Sinput}
\begin{Soutput}
[1] 0.7925003
\end{Soutput}
\end{Schunk}
\end{solution}

\noindent\textbf{\underline{Estimator performance metrics}} - Data Structure 0 (refer back to \texttt{R} Lab 4 for bias, variance, and MSE definitions).

\begin{enumerate}
\item Set the number of iterations \texttt{B} to 5 (to start).
\item Create a matrix \texttt{estimates\_data0} with \texttt{B} rows and \texttt{3} columns. Name the columns of the matrix \texttt{Param, ICE, TMLE}.
\item Within a for loop from \texttt{b} to \texttt{1:B}, do the following:
\begin{enumerate}
\item Redraw $n$ copies of the data using the \texttt{generate\_data0()} function you loaded earlier, and set equal to the object \texttt{ObsData0}. 
\item Implement the parametric g-computation, ICE g-computation, and TMLE using the functions \texttt{param.gcomp0\_fun()}, \texttt{ICE.gcomp0\_fun()}, and \texttt{TMLE.gcomp0\_fun()}, respectively, to estimate the causal parameter of interest, making sure you have specified the correct \texttt{abar} vector as an argument for each.
\item Save the estimates in the $b^{th}$ row of the \texttt{estimates\_data0} matrix.
\end{enumerate}
\item When you are confident that your code is working, increase the number of iterations \texttt{B = 500} and rerun your code. \textit{Warning: this may take a long time!}
\item For each estimator, estimate the:
\begin{itemize}
\item[-] Bias. \textit{Hint:} use the \texttt{colMeans()} function.
\item[-] Variance. \textit{Hint:} use the \texttt{var()} function on the estimates to get the covariance matrix, and take the diagonal of that matrix using the \texttt{diag()} function to get each estimator's variance.
\item[-] MSE. \textit{Hint:} use the \texttt{colMeans()} function.
\end{itemize}
\end{enumerate}


\begin{solution}
\begin{Schunk}
\begin{Sinput}
> # number of times to run
> B = 500
\end{Sinput}
\end{Schunk}
\begin{Schunk}
\begin{Sinput}
> # initialize matrix of estimates with column names
> estimates_data0 = matrix(NA, nrow = B, ncol = 3)
> colnames(estimates_data0) = c("Param", "ICE", "TMLE")
\end{Sinput}
\end{Schunk}
\begin{Schunk}
\begin{Sinput}
> # for loop to generate estimates
> for(b in 1:B){
+   
+   ObsData0 = generate_data0(n)
+   
+   estimates_data0[b,"Param"] = param.gcomp0_fun(abar = c(1,1))
+   estimates_data0[b,"ICE"] = ICE.gcomp0_fun(abar = c(1,1)) 
+   estimates_data0[b,"TMLE"] = TMLE.gcomp0_fun(abar = c(1,1))
+   
+ }
\end{Sinput}
\end{Schunk}
\begin{Schunk}
\begin{Sinput}
> # Bias
> colMeans(estimates_data0 - Psi.F0)
\end{Sinput}
\begin{Soutput}
       Param          ICE         TMLE 
 0.001044000 -0.001160395  0.001178556 
\end{Soutput}
\begin{Sinput}
> # Variance
> diag(var(estimates_data0))
\end{Sinput}
\begin{Soutput}
       Param          ICE         TMLE 
0.0003821573 0.0003804318 0.0004983707 
\end{Soutput}
\begin{Sinput}
> # MSE
> colMeans((estimates_data0 -Psi.F0)^2)
\end{Sinput}
\begin{Soutput}
       Param          ICE         TMLE 
0.0003824829 0.0003810174 0.0004987629 
\end{Soutput}
\end{Schunk}
\end{solution}

\pagebreak
\noindent\large\textbf{Data Structure 2: $O = (L(1), A(1), L(2), A(2), L(3), A(3), L(4), A(4), Y)$}
\normalsize


We are interested in the difference in the expected test score if all students got 8 or more hours of sleep for all 4 nights before the test versus if all students got less than 8 hours of sleep for all 4 nights before the test is 13.6183 points.
\[
\Psi^F(P_{U,X}) = E_{U,X}[Y_{\bar{a}(4)=1} - Y_{\bar{a}(4)=0}] = 13.6183
\]

\noindent Before we get started with estimation...
\begin{enumerate}
\item Load \texttt{DataStructure2.RData} using the \texttt{load()} function. Make sure you have specified the correct file path. You should see 6 new things come up in your global environment:
  
  \begin{packed_item}
\item[-] \texttt{ObsData2} - this is a dataframe of 1,000 observations that follows Data Structure 2 from previous labs.
\item[-] \texttt{Psi.F2} - this is the true $\Psi^F(P_{U,X})$ value for the target causal parameter $E_{U,X}[Y_{\bar{a}(4)=1}] - E_{U,X}[Y_{\bar{a}(4)=0}]$ (generated in lab 2).
\item[-] \texttt{generate\_data2} - this is the function that generates $n$ copies of Data Structure 2.
\item[-] \texttt{generate\_data2\_intervene}, \texttt{TrueMSMbeta1}, \texttt{TrueMSMbeta1\_wts} - we won't be using these objects, so you can remove them from your global environment if you'd like.
\end{packed_item}

\item Assign the number of students to \texttt{n}.
\item Create the function \texttt{rescale0to1()}, a function that takes as input a vector  of numbers \texttt{Y} and returns the numbers rescaled between 0 and 1:
\begin{Schunk}
\begin{Sinput}
> rescale0to1 = function(Y) {
+   rescaleY = (Y - min(Y))/(max(Y) - min(Y))
+   return(rescaleY)
+ }
\end{Sinput}
\end{Schunk}
We will apply this function on our outcome, $Y$, in the TMLE sections.
\end{enumerate}

\begin{solution}

\begin{Schunk}
\begin{Sinput}
> # set n to number of students
> n = nrow(ObsData2)
\end{Sinput}
\end{Schunk}
\begin{Schunk}
\begin{Sinput}
> # function that transforms a variable Y to be between 0 and 1
> rescale0to1 = function(Y) {
+   rescaleY = (Y - min(Y))/(max(Y) - min(Y))
+   return(rescaleY)
+ }
\end{Sinput}
\end{Schunk}
\end{solution}

%%%%% PARAMETRIC GCOMP OBSDATA2 %%%%%%%%
\noindent \textbf{\underline{Parametric g-computation}} - Data Structure 2
\begin{enumerate}
\item Create a new function \texttt{param.gcomp2\_fun()} that takes in \texttt{abar} as an argument. Within the function:
  \begin{enumerate}
\item First, estimate the conditional distributions of all the non-intervention nodes, in this case, $L(1)$, $L(2)$, $L(3)$, $L(4)$ and $Y$. For this example, we need estimates of:
  \begin{enumerate}
\item $\bar{Q}(Y|\bar{L}(4), \bar{A}(4))$, or the conditional expectation of $Y$, given the past. 
\item $\bar{Q}(L(t)|\bar{L}(t-1), \bar{A}(t-1))$, or the conditional expectation of $L(t)$, given the past, for $t = 2, 3, 4$.
\item $\bar{Q}(L(1))$, the baseline covariate distribution. To estimate the distribution of $L(1)$, we can use the empirical distribution. \emph{Hint:} go to the next step!
  \end{enumerate}
\item Use these estimates to generate (using Monte Carlo simulation) many ``counterfactual" covariate and outcome histories over time, setting $A(t) = a(t)$ for $t = 1,...,4$.  
\begin{enumerate}
\item Set \texttt{S} equal to $10,000$, the number of times we will simulate.
\item Sample \texttt{S} observations/rows, with replacement, from \texttt{ObsData2}.
\item Draw $L_i(1) = l_i(1)$ for each individual. That is, set \texttt{l1} equal to \texttt{L1} that lives within the simulated data.
\item Draw $L_i(2) = l_i(2)$ for each individual. 
\begin{enumerate}
\item From the estimated conditional distribution of $Q_n(L(2)| L(1), A(1))$, predict the expectation of $L(2)$, setting baseline exposure and covariate to $a(1)$ and $l(1)$, respectively. 
That is:
\begin{Schunk}
\begin{Sinput}
> Q.L2 = predict(Q.L2.reg, 
+                newdata = data.frame(L1 = l1, 
+                                     A1 = abar[1]))
\end{Sinput}
\end{Schunk}
\item Because $L(2)$ is continuous, for each individual, draw an observation from a $Normal(\mu_i = \bar{Q}_n(L_i(2) | L_i(1) = l_i(1), A_i(1) = a_i(1)), \sigma_i = \hat{\sigma}_{L(2)})$ distribution.
\begin{Schunk}
\begin{Sinput}
> l2 = rnorm(S, mean = Q.L2, sd = sd(ObsData2$L2))
\end{Sinput}
\end{Schunk}
\end{enumerate}
\item Draw $L_i(3) = l_i(3)$ for each individual.
\begin{enumerate}
\item From the estimated conditional distribution of $Q_n(L(3)| \bar{L}(2), \bar{A}(2))$, predict the expectation of $L(3)$, setting baseline exposure and covariate to $\bar{a}(2)$ and $\bar{l}(2)$, respectively. 
\item Because $L(3)$ is continuous, for each individual, draw an observation from a $Normal(\mu_i = \bar{Q}(L_i(3) | \bar{L}_i(2) = \bar{l}_i(2), \bar{A}_i(2) = \bar{a}_i(2)), \sigma_i = \hat{\sigma}_{L(3)})$ distribution.
\end{enumerate}
\item Draw $L_i(4) = l_i(4)$ and $Y_i = y_i$ for each individual, extending the steps used to draw $l_i(2)$ and $l_i(3)$ above.
\end{enumerate}
\item Estimate the expectation of $Y$ with the empirical mean of \texttt{y}. Have the function \texttt{param.gcomp2\_fun()} return this value.
\end{enumerate}
\item Estimate $E_{U,X}[Y_{\bar{a} = 1} - Y_{\bar{a} = 0}]$ using parametric g-computation by applying \texttt{param.gcomp0\_fun()}, with the correct \texttt{abar} argument specifications.
\end{enumerate}


\begin{solution}

\begin{Schunk}
\begin{Sinput}
> ### Estimation of parametric g-computation - ObsData2 ###
> 
> param.gcomp2_fun = function(abar){
+   
+   # Estimate conditional dist of non-intervention nodes #
+   # estimate Q(L(2) | L(1), A(1))
+   Q.L2.reg = glm(L2 ~ L1 + A1, data = ObsData2)
+   # estimate Q(L(3) | Lbar(2), Abar(2))
+   Q.L3.reg = glm(L3 ~ L1 + A1 + L2 + A2, data = ObsData2)
+   # estimate Q(L(4) | Lbar(3), Abar(3))
+   Q.L4.reg = glm(L4 ~ L1 + A1 + L2 + A2 + L3 + A3, data = ObsData2)
+ 
+   # estimate Q(Y | Lbar(4), Abar(4))
+   Q.Y.reg = glm(Y ~ ., data = ObsData2)
+   
+   # number of times to MC simulate
+   S = 10000
+   
+   # sample students/rows with replacement, S times
+   ObsData2.MC = ObsData2[sample(1:n, S, replace = T),]
+   
+   # draw L1 for each individual
+   l1 = ObsData2.MC$L1
+   
+   # draw L2 for each individual
+   Q.L2 = predict(Q.L2.reg, newdata = data.frame(L1 = l1, A1 = abar[1]))
+   l2 = rnorm(S, mean = Q.L2, sd = sd(ObsData2$L2))
+   
+   # draw L3 for each individual
+   Q.L3 = predict(Q.L3.reg, newdata = data.frame(L1 = l1,
+                                                 L2 = l2,
+                                                 A1 = abar[1],
+                                                 A2 = abar[2]))
+   l3 = rnorm(S, mean = Q.L3, sd = sd(ObsData2$L3))
+   
+   # draw L4 for each individual
+   Q.L4 = predict(Q.L4.reg, newdata = data.frame(L1 = l1,
+                                                 L2 = l2,
+                                                 L3 = l3,
+                                                 A1 = abar[1],
+                                                 A2 = abar[2],
+                                                 A3 = abar[3]))
+   l4 = rnorm(S, mean = Q.L4, sd = sd(ObsData2$L4))
+   
+   # draw Y for each individual
+   Q.Y = predict(Q.Y.reg, newdata = data.frame(L1 = l1,
+                                               L2 = l2,
+                                               L3 = l3,
+                                               L4 = l4,
+                                               A1 = abar[1], 
+                                               A2 = abar[2], 
+                                               A3 = abar[3],
+                                               A4 = abar[4]))
+   y = rnorm(S, mean = Q.Y, sd = sd(ObsData2$Y))
+   
+   # mean of simulated Ys is estimate
+   meanY = mean(y)
+   
+   return(meanY)
+ 
+ }
\end{Sinput}
\end{Schunk}

\begin{Schunk}
\begin{Sinput}
> Psi.hat.pgcomp2 = param.gcomp2_fun(abar = c(1,1,1,1)) - param.gcomp2_fun(abar = c(0,0,0,0))
> Psi.hat.pgcomp2
\end{Sinput}
\begin{Soutput}
[1] 13.61381
\end{Soutput}
\end{Schunk}

\end{solution}

%%%%% ICE OBSDATA2 %%%%%%%%
\noindent\textbf{\underline{ICE g-computation}} - Data Structure 2

\begin{enumerate}
\item Create a new function \texttt{ICE.gcomp2\_fun()} that takes in \texttt{abar} as an argument. Within the function:
\begin{enumerate}
\item Create a dataframe called \texttt{newdata} where $A(t)$ has been set to $a(t)$ for $t = 1,...,4$:
\begin{Schunk}
\begin{Sinput}
> newdata = ObsData2
> newdata$A1 = abar[1]
> newdata$A2 = abar[2]
> newdata$A3 = abar[3]
> newdata$A4 = abar[4]
\end{Sinput}
\end{Schunk}
\item Rescale $Y$ from \texttt{ObsData2} to be between 0 and 1 using the \texttt{rescale0to1()} function, and set it equal to the variable \texttt{Y.scaled}.
\item For time $t = 5$:
\begin{enumerate}
\item Regress rescaled $Y$ on \textit{observed history} at time 5 using \texttt{glm()}.
\item Generate predicted values of rescaled $Y$, evaluated at each observed covariate value and exposure history of interest (i.e., $\bar{L}_i(4) = \bar{l}_i(4)$ and $\bar{A}_i(4) = \bar{a}(4)$, respectively). Set the predicted values equal to \texttt{Q5}.
\end{enumerate}
\item For time $t = 4, 3, 2$:
\begin{enumerate}
\item Using the predicted values from the prior step as a new outcome, regress the new outcome on the \textit{observed past} at time $t$.
\item Generate new predicted values, evaluated \textit{at each observed covariate value and exposure history of interest} (i.e., $\bar{L}_i(t-1) = \bar{l}_i(t-1)$ and $\bar{A}_i(t-1) = \bar{a}(t-1)$, respectively).
\end{enumerate}
\item Take the empirical mean of the final predicted outcome. This is the value that \texttt{ICE.gcomp2\_fun()} should return.
\end{enumerate}
\item Estimate $E_{U,X}[Y_{\bar{a} = 1} - Y_{\bar{a} = 0}]$ using parametric g-computation by applying \texttt{ICE.gcomp2\_fun()}, with the correct \texttt{abar} argument specification.
\end{enumerate}

\begin{solution}
\begin{Schunk}
\begin{Sinput}
> ### Estimation of ICE g-computation - ObsData2 ###
> ICE.gcomp2_fun = function(abar){
+   
+   newdata = ObsData2
+   newdata$A1 = abar[1]
+   newdata$A2 = abar[2]
+   newdata$A3 = abar[3]
+   newdata$A4 = abar[4]
+   
+   # rescale Y to be between 0 and 1
+   Y.scaled = rescale0to1(ObsData2$Y)
+   
+   # regress Y on observed past
+   Q5.reg = glm(Y.scaled ~ A1 + A2 + A3 + A4 + L1 + L2 + L3 + L4, data = ObsData2, family = "quasibinomial")
+   
+   # generate predicted values of Y at exposure history we want = Q5
+   Q5 = predict(Q5.reg, newdata = newdata, type = "response")
+   
+   # regress Q5 on observed past
+   Q4.reg = glm(Q5 ~ A1 + A2 + A3 + L1 + L2 + L3, data = ObsData2, family = "quasibinomial")
+   
+   # generate predicted values at exposure history we want = Q4 
+   Q4 = predict(Q4.reg, newdata = newdata, type = "response")
+   
+   # regress Q4 on observed past
+   Q3.reg = glm(Q4 ~ A1 + A2 + L1 + L2, data = ObsData2, family = "quasibinomial")
+   
+   # generate predicted values at exposure levels we want = Q3
+   Q3 = predict(Q3.reg, newdata = newdata, type = "response")
+   
+   # regress Q3 on observed past
+   Q2.reg = glm(Q3 ~ A1 + L1, data = ObsData2, family = "quasibinomial")
+   
+   # generate predicted values at exposure levels we want = Q2
+   Q2 = predict(Q2.reg, newdata = newdata, type = "response")
+   
+   # mean of Ypred4 is estimate
+   meanQ2 = mean(Q2)
+   
+   # back-transform estimate to original scale of Y
+   meanQ.back = meanQ2*(max(ObsData2$Y) - min(ObsData2$Y)) + min(ObsData2$Y)
+   
+   return(meanQ.back)
+   
+ }
\end{Sinput}
\end{Schunk}
\begin{Schunk}
\begin{Sinput}
> Psi.hat.ICE2 = ICE.gcomp2_fun(abar = c(1,1,1,1)) - ICE.gcomp2_fun(abar = c(0,0,0,0))
> Psi.hat.ICE2
\end{Sinput}
\begin{Soutput}
[1] 13.38402
\end{Soutput}
\end{Schunk}
\end{solution}

%%%%% TMLE OBSDATA2 %%%%%%%%

\noindent \textbf{\underline{TMLE}} - Data Structure 2
\begin{enumerate}
\item Create a new function \texttt{TMLE.gcomp2\_fun()} that takes in \texttt{abar} as an argument. Within the function:
\begin{enumerate}
\item Create a dataframe called \texttt{newdata} where $A(t)$ has been set to $a(t)$ for $t = 1,...,4$.
\item Estimate the probability of each person's observed exposure at time $t$, i.e., $P_0(A(t)|\bar{L}(t), \bar{A}(t-1)) = g_0(A(t)|\bar{L}(t), \bar{A}(t-1))$ for $t = 1,...,4$ 
\begin{enumerate}
\item First, estimate the probability of receiving treatment at time $t$ $P_0(A(t)=1|\bar{L}(t), \bar{A}(t-1)) = g_0(A(t)=1|\bar{L}(t), \bar{A}(t-1))$ for $t = 1,...,4$ using correctly specified parametric regression models. The correct model specifications are (refer back to \texttt{R} lab 3 for reference):
\begin{align*}
g_0(A(1)=1|L(1)) & = expit[\beta_0 + \beta_1L(1)] \\
g_0(A(2)=1|\bar{L}(2), A(1)) & = expit[\beta_0 + \beta_1L(1) + \beta_2A(1) + \beta_3L(2)] \\
g_0(A(3)=1|\bar{L}(3), \bar{A}(2)) & = expit[\beta_0 + \beta_1L(1) + \beta_2A(1) + \beta_3L(2) + \beta_4A(2) + \beta_5L(3)] \\
g_0(A(4)=1|\bar{L}(4), \bar{A}(3)) & = expit[\beta_0 + \beta_1L(1) + \beta_2A(1) + \beta_3L(2) + \beta_4A(2) + \beta_5L(3) + \beta_6A(3) + \beta_7L(4)]
\end{align*}
\item Use the \texttt{glm()} function, and specify the arguments \texttt{family = 'binomial'} for logistic regression and \texttt{data = ObsData2}. 
\item Predict each subject's probability of the exposure at time $t$, given their observed exposure and observed covariate history, i.e., $g_n(A_i(t)=1|\bar{A}_i(t-1), \bar{L}_i(t))$, for $t = 1,...,4$.
\item Obtain the conditional probabilities of each subject's observed exposure, $g_n(A_i(t) | \bar{A}_i(t-1), \bar{L}_i(t))$. That is, for each timepoint ($t = 1,...,4$):
\begin{itemize}
\item[-] Among subjects who got $A(t)=1$ at time $t$, assign the predicted probability as $g_n(A_i(t) = 1| \bar{A}_i(t-1), \bar{L}_i(t))$. 
\item[-] Similarly, among subjects who got $A(t) = 0$ at time $t$, assign the predicted probability as $g_n(A_i(t) = 0| \bar{A}_i(t-1), \bar{L}_i(t))$.
\end{itemize}
For example, for timepoint 1:
\begin{Schunk}
\begin{Sinput}
> g.abar1 = (ObsData2$A1 == 1) * g.abar1.1 + (ObsData2$A1 == 0) * (1 - g.abar1.1)
\end{Sinput}
\end{Schunk}
Repeat for $t = 2, 3, 4$.
\end{enumerate}
\item For each timepoint $t = 1,...,4$, create a logical variable that indicates which students had a treatment history $\bar{A}(t) = \bar{a}(t)$. For example for $t = 1$:
\begin{Schunk}
\begin{Sinput}
> I1 = ObsData2$A1 == abar[1] 
\end{Sinput}
\end{Schunk}
Do the same for $t = 2, 3, 4$.
\item For $t = 5$:
\begin{enumerate}
\item Rescale $Y$ from \texttt{ObsData2} to be between 0 and 1 using the \texttt{rescale0to1()} function, and set it equal to the variable \texttt{Y.scaled}.
\item Estimate $\bar{Q}^0_{5,n}(\bar{a}(4), \bar{L}(4))$. This is the conditional expectation of $Y$ (i.e., \texttt{Y.scaled}), given the past covariates and exposure of interest.
\begin{enumerate}
\item Regress rescaled $Y$ on the \textit{observed past}, $\bar{A}(4)$ and $\bar{L}(4)$:
\begin{Schunk}
\begin{Sinput}
> Q5.reg = glm(Y.scaled ~ L1 + A1 + L2 + A2 + L3 + A3 + L4 + A4, 
+              data = ObsData2, family = "quasibinomial")
\end{Sinput}
\end{Schunk}
\item Using the above model, predict $logit(\bar{Q}_{5,n})$, the conditional outcome for all subjects on the logit scale \textit{at the exposure history we want} (i.e., $\bar{A}(4) = \bar{a}(4)$). Call this vector \texttt{logit.Q5}.
\begin{Schunk}
\begin{Sinput}
> logit.Q5 = predict(Q5.reg, type = "link", newdata = newdata)
\end{Sinput}
\end{Schunk}
\end{enumerate}
\item Update $\bar{Q}^0_{5,n}(\bar{a}(4), \bar{L}(4))$ to $\bar{Q}^\star_{5,n}(\bar{a}(4), \bar{L}(4))$
\begin{enumerate}
\item Make the clever covariate, $H_{n,5}(\bar{A}(4), \bar{L}(4)) = \frac{\mathbb{I}[\bar{A}(4) = \bar{a}(4)]}{\prod_{t=1}^4g_n(A(t)|\bar{L}(t), \bar{A}(t-1))}$, where the denominator $\prod_{t=1}^4g_n(A(t)|\bar{L}(t), \bar{A}(t-1))$ is bounded between 0.01 and 1:
\begin{Schunk}
\begin{Sinput}
> H5 = I4/bound(g.abar1 * g.abar2 * g.abar3 * g.abar4)
\end{Sinput}
\end{Schunk}
\item Fit a logistic regression of Y on the intercept, with $logit(\bar{Q}^0_{5,n})$ as an offset and clever covariate $H_{n,5}(\bar{A}(4), \bar{L}(4))$ as weight:
\begin{Schunk}
\begin{Sinput}
> Q5.reg.update = glm(Y.scaled ~ offset(logit.Q5), weights = H5, 
+                     family = "quasibinomial")
\end{Sinput}
\end{Schunk}
\item Generate $\bar{Q}^\star_{5,n}$, the predicted probabilities of the updated model in the previous step:
\begin{Schunk}
\begin{Sinput}
> Q5.star = predict(Q5.reg.update, type = "response")
\end{Sinput}
\end{Schunk}
\end{enumerate}
\end{enumerate}
\item For $t = 4$:
\begin{enumerate}
\item Estimate $\bar{Q}^0_{4,n}(\bar{a}(3), \bar{L}(3))$. This is the conditional expectation of $\bar{Q}^\star_{5,n}$ (i.e., \texttt{Q5.star}), given the past covariates and exposure of interest.
\begin{enumerate}
\item Regress $\bar{Q}^\star_{5,n}$ on the \textit{observed past}, $\bar{A}(3)$ and $\bar{L}(3)$:
\begin{Schunk}
\begin{Sinput}
> Q4.reg = glm(Q5.star ~ L1 + A1 + L2 + A2 + L3 + A3, 
+              data = ObsData2, family = "quasibinomial")
\end{Sinput}
\end{Schunk}
\item Using the above model, predict $logit(\bar{Q}_{4,n})$, the conditional outcome for all subjects \textit{at exposure history we want} (i.e., $\bar{A}(3) = \bar{a}(3)$). Call this vector \texttt{logit.Q4}.
\begin{Schunk}
\begin{Sinput}
> logit.Q4 = predict(Q4.reg, newdata = newdata, type = "link")
\end{Sinput}
\end{Schunk}
\end{enumerate}
\item Update $\bar{Q}^0_{4,n}(\bar{a}(3), \bar{L}(3))$ to $\bar{Q}^\star_{4,n}(\bar{a}(3), \bar{L}(3))$
\begin{enumerate}
\item Make the clever covariate, $H_{n,4}(\bar{A}(3), \bar{L}(3)) = \frac{\mathbb{I}[\bar{A}(3) = \bar{a}(3)]}{\prod_{t=1}^3g_n(A(t)|\bar{L}(t), \bar{A}(t-1))}$, where the denominator $\prod_{t=1}^3g_n(A(t)|\bar{L}(t), \bar{A}(t-1))$ is bounded between 0.01 and 1:
\begin{Schunk}
\begin{Sinput}
> H4 = I3/bound(g.abar1 * g.abar2 * g.abar3)
\end{Sinput}
\end{Schunk}
\item Fit a logistic regression of $\bar{Q}^\star_{5,n}$ on the intercept, with $logit(\bar{Q}^0_{4,n})$ as an offset and clever covariate $H_{n,4}(\bar{A}(3), \bar{L}(3))$ as weight:
\begin{Schunk}
\begin{Sinput}
> Q4.reg.update = glm(Q5.star ~ offset(logit.Q4), weights = H4, family = "quasibinomial")
\end{Sinput}
\end{Schunk}
\item Generate $\bar{Q}^\star_{4,n}$, the predicted probabilities of the updated model in the previous step:
\begin{Schunk}
\begin{Sinput}
> Q4.star = predict(Q4.reg.update, type = "response")
\end{Sinput}
\end{Schunk}
\end{enumerate}
\end{enumerate}
\item For $t = 3, 2$:
\begin{enumerate}
\item Estimate $\bar{Q}^0_{t,n}(\bar{a}(t-1), \bar{L}(t-1))$. This is the conditional expectation of $\bar{Q}^\star_{t+1,n}$, given the past covariates and exposure of interest.
\begin{enumerate}
\item Regress $\bar{Q}^\star_{t+1,n}$ on the \textit{observed past}, $\bar{A}(t-1)$ and $\bar{L}(t-1)$
\item Using the above model, predict $logit(\bar{Q}_{t,n})$, the conditional outcome on the logit scale for all subjects \textit{at exposure history we want} (i.e., $\bar{A}(t-1) = \bar{a}(t-1)$)
\end{enumerate}
\item Update $\bar{Q}^0_{t,n}(\bar{a}(t-1), \bar{L}(t-1))$ to $\bar{Q}^\star_{t,n}(\bar{a}(t-1), \bar{L}(t-1))$
\begin{enumerate}
\item Make the clever covariate, $H_{n,t}(\bar{A}(t-1), \bar{L}(t-1)) = \frac{\mathbb{I}[\bar{A}(t-1) = \bar{a}(t-1)]}{\prod_{j=1}^{t-1}g_n(A(j)|\bar{L}(j), \bar{A}(j-1))}$
\item Fit a logistic regression of $\bar{Q}^\star_{t+1,n}$ on the intercept, with $logit(\bar{Q}^0_{t,n})$ as an offset and clever covariate $H_{n,t}(\bar{A}(t-1), \bar{L}(t-1))$ as weight.
\item Generate $\bar{Q}^\star_{t,n}$, the predicted probabilities of the updated model in the previous step.
\end{enumerate}
\end{enumerate}
\item Take the mean of $\bar{Q}^\star_{2,n}$ and back-transform this number to the original scale of $Y$. For example, if \texttt{meanQ2.star} is the mean of $\bar{Q}^\star_{2,n}$, this would be the code to back-transform the estimate:
\begin{Schunk}
\begin{Sinput}
> meanQ.star.back = meanQ2.star*(max(ObsData2$Y) - min(ObsData2$Y)) + min(ObsData0$Y)
\end{Sinput}
\end{Schunk}
This is your TMLE estimate! Have \texttt{TMLE.gcomp2\_fun()} return this value.
\end{enumerate}
\item Estimate $E_{U,X}[Y_{\bar{a} = 1} - Y_{\bar{a} = 0}]$ using TMLE by applying \texttt{TMLE.gcomp2\_fun()}, with the correct \texttt{abar} argument specifications.
\end{enumerate}

\begin{solution}
\begin{Schunk}
\begin{Sinput}
> TMLE.gcomp2_fun = function(abar){
+   
+   # create newdata
+   newdata = ObsData2
+   newdata$A1 = abar[1]
+   newdata$A2 = abar[2]
+   newdata$A3 = abar[3]
+   newdata$A4 = abar[4]
+   
+   # estimate treatment mechanism at each point
+   gA1.reg = glm(A1 ~ L1, family = "binomial", data = ObsData2)
+   gA2.reg = glm(A2 ~ L1 + A1 + L2, family = "binomial", data = ObsData2)
+   gA3.reg = glm(A3 ~ L1 + A1 + L2 + A2 + L3, family = "binomial", data = ObsData2)
+   gA4.reg = glm(A4 ~ L1 + A1 + L2 + A2 + L3 + A3 + L4, family = "binomial", data = ObsData2)
+   
+   # with above model, compute predicted probability of getting 8 hrs of 
+   # sleep at time t given observed past
+   g.abar1.1 = predict(gA1.reg, type = "response")
+   g.abar2.1 = predict(gA2.reg, type = "response")
+   g.abar3.1 = predict(gA3.reg, type = "response")
+   g.abar4.1 = predict(gA4.reg, type = "response")
+   
+   # compute predicted probability of observed exposure A(t), given history
+   g.abar1 = g.abar2 = g.abar3 = g.abar4 = rep(NA, length = n)
+   g.abar1 = (ObsData2$A1 == 1) * g.abar1.1 + (ObsData2$A1 == 0) * (1 - g.abar1.1)
+   g.abar2 = (ObsData2$A2 == 1) * g.abar2.1 + (ObsData2$A2 == 0) * (1 - g.abar2.1)
+   g.abar3 = (ObsData2$A3 == 1) * g.abar3.1 + (ObsData2$A3 == 0) * (1 - g.abar3.1)
+   g.abar4 = (ObsData2$A4 == 1) * g.abar4.1 + (ObsData2$A4 == 0) * (1 - g.abar4.1)
+ 
+   # indicator of each abar(t) = 1 for t = 1,...4
+   I4 = ObsData2$A1 == abar[1] & ObsData2$A2 == abar[2] & ObsData2$A3 == abar[3] & ObsData2$A4 == abar[4]
+   I3 = ObsData2$A1 == abar[1] & ObsData2$A2 == abar[2] & ObsData2$A3 == abar[3]
+   I2 = ObsData2$A1 == abar[1] & ObsData2$A2 == abar[2]
+   I1 = ObsData2$A1 == abar[1]
+   
+   # rescale Y to be between 0 and Y
+   Y.scaled = rescale0to1(ObsData2$Y)
+   
+   ### t = 5 ###
+   # fit model for E[Q5|Lbar(4), Abar(4)] 
+   Q5.reg = glm(Y.scaled ~ L1 + A1 + L2 + A2 + L3 + A3 + L4 + A4, data = ObsData2, family = "quasibinomial") 
+   # predict conditional outcome at exposure history we want on logit scale
+   logit.Q5 = predict(Q5.reg, newdata = newdata, type = "link") 
+   # make clever covariate
+   H5 = I4/ bound(g.abar1 * g.abar2 * g.abar3 * g.abar4)
+   # update E[Y|Lbar(4), abar(4)] by running regression with logit(Q5) as offset 
+   # and clever covariate H5 as weight
+   Q5.reg.update = glm(Y.scaled ~ offset(logit.Q5), weights = H5, family = "quasibinomial")
+   # update the expected value of Q
+   Q5.star = predict(Q5.reg.update, type = "response")
+   
+   ### t = 4 ###
+   Q4.reg = glm(Q5.star ~ L1 + A1 + L2 + A2 + L3 + A3, data = ObsData2, family = "quasibinomial") 
+   logit.Q4 = predict(Q4.reg, newdata = newdata, type = "link") 
+   H4 = I3 / bound(g.abar1 * g.abar2 * g.abar3)
+   Q4.reg.update = glm(Q5.star ~ offset(logit.Q4), weights = H4, family = "quasibinomial")
+   Q4.star = predict(Q4.reg.update, type = "response")
+   
+   ### t = 3 ###
+   Q3.reg = glm(Q4.star ~ L1 + A1 + L2 + A2, data = ObsData2, family = "quasibinomial") 
+   logit.Q3 = predict(Q3.reg, newdata = newdata, type = "link") 
+   H3 = I2 / bound(g.abar1 * g.abar2)
+   Q3.reg.update = glm(Q4.star ~ offset(logit.Q3), weights = H3, family = "quasibinomial")
+   Q3.star = predict(Q3.reg.update, type = "response")
+   
+   ### t = 2 ###
+   Q2.reg = glm(Q3.star ~ L1 + A1, data = ObsData2, family = "quasibinomial") 
+   logit.Q2 = predict(Q2.reg, newdata = newdata, type = "link") 
+   H2 = I1 / bound(g.abar1)
+   Q2.reg.update = glm(Q3.star ~ offset(logit.Q2), weights = H2, family = "quasibinomial")
+   Q2.star = predict(Q2.reg.update, type = "response")
+   
+   # average of Q2.star is estimate
+   meanQ2.star = mean(Q2.star)
+   
+   # back-transform estimate to original scale of Y
+   meanQ.star.back = meanQ2.star*(max(ObsData2$Y) - min(ObsData2$Y)) + min(ObsData2$Y)
+   
+   return(meanQ.star.back)
+   
+ }
> 
\end{Sinput}
\end{Schunk}
\begin{Schunk}
\begin{Sinput}
> Psi.hat.TMLE2 = TMLE.gcomp2_fun(abar = c(1,1,1,1)) - TMLE.gcomp2_fun(abar = c(0,0,0,0))
> Psi.hat.TMLE2
\end{Sinput}
\begin{Soutput}
[1] 13.25996
\end{Soutput}
\end{Schunk}

\end{solution}

\noindent\textbf{\underline{Estimator performance metrics}} - Data Structure 2 (refer back to \texttt{R} Lab 4 for bias, variance, and MSE definitions).

\begin{enumerate}
\item Set the number of iterations \texttt{B} to 5 (to start).
\item Create a matrix \texttt{estimates\_data2} with \texttt{B} rows and \texttt{3} columns. Name the columns of the matrix \texttt{Param, ICE, TMLE}.
\item Within a for loop from \texttt{b} to \texttt{1:B}, do the following:
\begin{enumerate}
\item Redraw $n$ copies of the data using the \texttt{generate\_data2()} function you loaded earlier, and set equal to the object \texttt{ObsData2}. 
\item Implement the parametric g-computation, ICE g-computation, and TMLE using the functions \texttt{param.gcomp2\_fun()}, \texttt{ICE.gcomp2\_fun()}, and \texttt{TMLE.gcomp2\_fun()}, respectively, to estimate the causal parameter of interest, making sure you have specified the correct \texttt{abar} vector as an argument for each.
\item Save the estimates in the $b^{th}$ row of the \texttt{estimates\_data2} matrix.
\end{enumerate}
\item When you are confident that your code is working, increase the number of iterations \texttt{B = 500} and rerun your code. 
\item For each estimator, estimate the bias, variance, and MSE.
\end{enumerate}

\begin{solution}
\begin{Schunk}
\begin{Sinput}
> # number of times to run
> B = 500
\end{Sinput}
\end{Schunk}
\begin{Schunk}
\begin{Sinput}
> # initialize matrix of estimates with column names
> estimates_data2 = matrix(NA, nrow = B, ncol = 3)
> colnames(estimates_data2) = c("Param", "ICE", "TMLE")
\end{Sinput}
\end{Schunk}
\begin{Schunk}
\begin{Sinput}
> # for loop to generate estimates
> for(b in 1:B){
+   
+   ObsData2 = generate_data2(n)
+   
+   estimates_data2[b,"Param"] = param.gcomp2_fun(abar = c(1,1,1,1)) - param.gcomp2_fun(abar = c(0,0,0,0))
+   estimates_data2[b,"ICE"] = ICE.gcomp2_fun(abar = c(1,1,1,1)) - ICE.gcomp2_fun(abar = c(0,0,0,0))
+   estimates_data2[b,"TMLE"] = TMLE.gcomp2_fun(abar = c(1,1,1,1)) - TMLE.gcomp2_fun(abar = c(0,0,0,0))
+ }
> # Bias
> colMeans(estimates_data2 - Psi.F2)
\end{Sinput}
\begin{Soutput}
      Param         ICE        TMLE 
 0.02514181 -0.05169260 -0.02017198 
\end{Soutput}
\begin{Sinput}
> # Variance
> diag(var(estimates_data2))
\end{Sinput}
\begin{Soutput}
    Param       ICE      TMLE 
0.2534720 0.2235375 0.3907956 
\end{Soutput}
\begin{Sinput}
> # MSE
> #colMeans((estimates_data2 -Psi.F2)^2)
> (colMeans(estimates_data2 - Psi.F2))^2 + diag(var(estimates_data2))
\end{Sinput}
\begin{Soutput}
    Param       ICE      TMLE 
0.2541041 0.2262097 0.3912025 
\end{Soutput}
\begin{Sinput}
> 
\end{Sinput}
\end{Schunk}

\end{solution}

\pagebreak
\section{For Your Project: Estimation via forms of g-computation}

Think through the following questions and apply them to the dataset you will use for your final project.

\begin{enumerate}
\item Implement the parametric g-computation, ICE representation of g-computation, and TMLE estimators on your real data and simulated data.
\item For the simulated data only: obtain bias, variance, MSE for the three estimators based on the true value of the causal parameter you generated in \texttt{R} Lab 2.
\end{enumerate}


\pagebreak
\section{Optional Feedback}

You may attach responses to these questions to your lab. Thank you in advance!

\begin{enumerate}
\item Did you catch any errors in this lab? If so, where?
\item What did you learn in this lab?
\item Do you think that this lab met the goals listed at the beginning? 
\item What else would you have liked to review? What would have helped your understanding?
\item Any other feedback?
\end{enumerate}


\end{document}


